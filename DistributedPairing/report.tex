\documentclass[a4paper,11pt]{article}

\usepackage[T1]{fontenc}
\usepackage[utf8]{inputenc}
\usepackage{graphicx}
\usepackage{xcolor}

% --- Font Settings ---
\renewcommand\familydefault{\sfdefault}
\usepackage{tgheros}
% zi4 is a high-quality, professional monospaced font available in almost all distributions
\usepackage[varqu]{zi4} 

\usepackage{amsmath,amssymb,amsthm,textcomp}
\usepackage{geometry}
\geometry{left=25mm,right=25mm, top=20mm,bottom=20mm}

\linespread{1.3}
\newcommand{\linia}{\rule{\linewidth}{0.5pt}}

% --- Header and Footer ---
\usepackage{fancyhdr}
\pagestyle{fancy}
\lhead{}
\chead{}
\rhead{}
\lfoot{Assignment Report}
\cfoot{}
\rfoot{Page \thepage}
\renewcommand{\headrulewidth}{0pt}

% --- Code Listing Settings ---
\usepackage{listings}
\lstset{
    language=Go,
    basicstyle=\ttfamily\small,
    aboveskip={1.0\baselineskip},
    belowskip={1.0\baselineskip},
    columns=fixed,
    extendedchars=true,
    breaklines=true,
    tabsize=4,
    frame=lines,
    keywordstyle=\color[rgb]{0.627,0.126,0.941},
    commentstyle=\color[rgb]{0.133,0.545,0.133},
    stringstyle=\color[rgb]{0,0,0},
    numbers=left,
    numberstyle=\small,
    stepnumber=1,
    numbersep=10pt,
    captionpos=t
}

\begin{document}

\begin{center}
    \vspace{2ex}
    {\huge \textsc{Distributed Pairing}}
    \vspace{1ex}
    \\
    \linia\\
    Alessandro Monticelli, Jóhannes Helgi Tómasson, Jonas Stahl \hfill \today
    \vspace{4ex}
\end{center}

\section*{1. Description of the Solution}
% You describe the algorithm, use the proper communication methods, 
% and use the specified shared variables for the result.

\section*{2. Proposed Pseudocode}
\begin{lstlisting}[caption=Generalized 3-Process Mutual Exclusion]
// Enter your pseudocode here.
// Use the flag logic (Copier/Inverter) for Alice, Bob, and Charlie.
\end{lstlisting}

\section*{3. Argument of Correctness}
% You demonstrate that the no two neighbouring nodes are single 
% if the algorithm terminates

\section*{4. Termination}
% You give a convincing argument that your algorithm terminates. Reference the code.

\section*{5. Cooperation}
% You give a convincing algorithm that no node waits for messages that are never sent.
\end{document}
